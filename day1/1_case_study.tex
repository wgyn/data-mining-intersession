\documentclass{beamer}

\usepackage{amsmath}
\usepackage{graphicx}
\usepackage{multicol}

\setkeys{Gin}{width=.5\textwidth}

\title{Case study: data-driven pricing}
\author{Scott Powers and Ryan Wang}

\usepackage{Sweave}
\begin{document}
\input{1_case_study-concordance}

\begin{frame}
\titlepage
\end{frame}

\begin{frame}[fragile]{The problem}
\begin{itemize}
	\item A retail client wanted to improve their strategy for pricing apparel. 
	\item In the past, they would start with a base price (usually set by executives) and mark-down whatever wasn't sold towards the end of the season. It would typically look something like:
  \begin{figure}
    \includegraphics{figs/price_season.pdf}
  \end{figure}
\end{itemize}
\end{frame}

\begin{frame}{The problem cont.}
\begin{itemize}
	\item However, this heuristic pricing led to high variance in sell-through rates at the end of the season.
	\item Popular products would sell out too quickly, while unpopular products would never sell out and turn into excess inventory at the end of the season.
	\item A better strategy would have been to raise prices on popular products (or discount them less aggressively), and vice versa for unpopular products.
\end{itemize}
\end{frame}

\begin{frame}{The problem cont.}
\begin{itemize}
  \item A histogram of sell-through rates at the end of the season would look like:
  \begin{figure}
    \includegraphics[height=.8\textheight,keepaspectratio=true]{figs/sell_through.pdf}
  \end{figure}
\end{itemize}
\end{frame}

\begin{frame}{Using data to improve pricing}
\begin{itemize}
	\item A bit of economics can be used to create a model that maximizes profits over the entirety of the season.
	\item Roughly we'd like to maximize profit ($\Pi$) by changing price (P)...
	$$ max_P \Pi = PQ_s - CQ_0 $$
	\item ...where quantity sold is a function of price, demand (D), and elasticity ($\epsilon$), which is a fancy term for the responsiveness of buyers to price
	$$ Q_s = DP^{\epsilon} $$
  \item Using data to implement the model, we would take in weekly inputs such as last week's sales and time of year and output recommended prices
\end{itemize}
\end{frame}

\begin{frame}{Two data-mining tasks}
\begin{itemize}
  \item A common task in many fields (physics, economics, etc) is to estimate an equation from noisy data
  \item We observe combinations of price and quantity sold e.g. sold 10 dresses when pricing at \$10
  \item We could then fit a line to estimate $D$ and $\epsilon$, tracing out the price-quantity relationship and allowing for predictions.
  \begin{figure}
    \includegraphics[height=.6\textheight,keepaspectratio=true]{figs/pq.pdf}
  \end{figure}  
\end{itemize}
\end{frame}

\end{document}
