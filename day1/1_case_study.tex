\documentclass{beamer}

\usepackage{amsmath}
\usepackage{graphicx}
\usepackage{multicol}

\title{Case study: data-driven pricing}
\author{Scott Powers and Ryan Wang}

\begin{document}

\begin{frame}
\titlepage
\end{frame}

\begin{frame}{The problem}
\begin{itemize}
	\item A retail client wanted to improve their strategy for pricing apparel. 
	\item In the past, they would start with a base price (usually set by executives) and mark-down whatever wasn't sold towards the end of the season.
	\item As such, plotting price over time would usually give something like:
	\item TO DO: INCLUDE FAKE PLOT, PRICE VS WEEK, LINEARLY DECREASING, BIG DROP AT END
\end{itemize}
\end{frame}

\begin{frame}{The problem cont.}
\begin{itemize}
	\item However, this heuristic pricing led to high variance in sell-through rates at the end of the season:
	\item TO DO: INCLUDE FAKE PLOT, ALMOST UNIFORM DISTRIBUTION IN SELL-THROUGH (S19 OF APPAREL PRICING DECK)
	\item Popular products would sell out too quickly, while unpopular products would never sell out and turn into excess inventory at the end of the season.
	\item A better strategy would have been to raise prices on popular products (or discount them less aggressively), and vice versa for unpopular products.
\end{itemize}
\end{frame}

\begin{frame}{Using data to improve pricing}
\begin{itemize}
	\item On a weekly basis, the data-driven pricing model took in inputs such as last week's sales and the week (e.g. if it's the week of Thanksgiving) and outputted recommended prices
	\item The model roughly said to maximize profit ($\Pi$) by changing price (P)...
	$$ max_P \Pi = PQ_s - CQ_0 $$
	\item ...where quantity sold is a function of price, demand (D), and elasticity ($\epsilon$), which is a fancy term for the responsiveness of buyers to price
	$$ Q = DP^{\epsilon} $$ 
\end{itemize}
\end{frame}

\begin{frame}{Two data-mining tasks}
\begin{enumerate}
	\item Estimate elasticity, or how responsive buyers are to price
	\item TO DO: INCLUDE FAKE PLOT OF QUANTITY VERSUS PRICE WITH FITTED LINE
	\item Predict demand
	\item TO DO: SHOW A TIME SERIES OF DEMAND WITH DOTTED LINES TO INDICATE POSSIBLE 'FORECASTS'
\end{enumerate}
\end{frame}

\end{document}